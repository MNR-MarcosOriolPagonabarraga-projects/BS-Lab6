\section{Effect of Trials on ERP Averaging}

Synchronized averaging is essential for recovering ERP components from noisy EEG.
While the ERP waveform is deterministic across repetitions, the background EEG
activity is stochastic. Therefore, averaging $N$ trials increases the signal-to-noise
ratio (SNR), causing the ERP to emerge progressively from the noise. The noise
variance decreases proportionally to $1/\sqrt{N}$, which explains why larger
numbers of trials yield clearer and more stable waveforms.

\subsection*{2(a). Stimulus-Locked Averages (P2, N2, P3)}

For both congruent and incongruent stimuli, progressive averages were computed
using the first 10, 20, 30, and 40 trials, followed by the full average across all
available epochs.

The averaged ERP responses for each channel are shown below.

\begin{figure}[H]
    \centering
    \includegraphics[width=\textwidth]{img/AV-ERP-Fz.png}
    \caption{Averaged stimulus-locked ERP — Channel Fz. Clear emergence of P2, N2, and P3 components as the number of averaged trials increases.}
    \label{fig:av_erp_fz}
\end{figure}

\begin{figure}[H]
    \centering
    \includegraphics[width=\textwidth]{img/AV-ERP-Cz.png}
    \caption{Averaged stimulus-locked ERP — Channel Cz. ERP morphology becomes progressively stable, especially for the P3 component.}
    \label{fig:av_erp_cz}
\end{figure}

\begin{figure}[H]
    \centering
    \includegraphics[width=\textwidth]{img/AV-ERP-Pz.png}
    \caption{Averaged stimulus-locked ERP — Channel Pz. Parietal sites show the strongest P3 amplitude, consistent with classical ERP topography.}
    \label{fig:av_erp_pz}
\end{figure}

Figures for channels Fz, Cz, and Pz show that:

\begin{itemize}
    \item With only 10 trials, the waveform is dominated by noise and ERP
    components are difficult to identify.
    \item With 20-30 trials, the P2, N2, and P3 begin to emerge with consistent
    polarity and latency.
    \item With 40 trials and the full average, the morphology stabilizes and the ERP
    peaks appear clearly defined.
\end{itemize}

This behaviour is expected from ERP theory: components such as P3 (250–600 ms)
are highly sensitive to averaging because of their low amplitude relative to background EEG.

\subsection*{2(b). Response-Locked Averages (ERN)}

A parallel analysis was performed for response-locked epochs. The averaged ERN
waveforms for each channel are shown below.

\begin{figure}[H]
    \centering
    \includegraphics[width=\textwidth]{img/AV-ERN-Fz.png}
    \caption{Averaged response-locked ERN — Channel Fz. Clear frontal negative peak emerging as the number of trials increases.}
    \label{fig:av_ern_fz}
\end{figure}

\begin{figure}[H]
    \centering
    \includegraphics[width=\textwidth]{img/AV-ERN-Cz.png}
    \caption{Averaged response-locked ERN — Channel Cz. The ERN becomes more defined around 30–40 trials, showing the expected fronto-central distribution.}
    \label{fig:av_ern_cz}
\end{figure}

\begin{figure}[H]
    \centering
    \includegraphics[width=\textwidth]{img/AV-ERN-Pz.png}
    \caption{Averaged response-locked ERN — Channel Pz. Parietal regions show weaker ERN amplitude, consistent with its typical topography.}
    \label{fig:av_ern_pz}
\end{figure}

The ERN (0–100 ms after the response) shows:

\begin{itemize}
    \item High variability for small $N$, with several trials showing opposite
    polarity relative to the expected ERN.
    \item Clear negative deflection as $N$ increases, stabilizing around 30–40 trials.
    \item The full average produces a well-defined ERN morphology with the
    characteristic frontocentral distribution.
\end{itemize}

\subsection*{ Interpretation}

Across both stimulus-locked and response-locked analyses:

\begin{itemize}
    \item Increasing the number of trials significantly improves ERP SNR.
    \item The morphology, amplitude, and latency of P2, N2, P3, and ERN stabilize
    as $N$ increases.
    \item Beyond approximately 40 trials, improvements become marginal,
    indicating convergence toward a stable ERP estimate.
\end{itemize}

Thus, this exercise confirms the fundamental importance of synchronized averaging
for recovering reliable ERPs from noisy EEG recordings.
