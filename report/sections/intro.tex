\section*{Introduction}
The analysis of \textbf{Event-Related Potentials (ERPs)} is a cornerstone of non-invasive 
neurophysiology, utilizing time-locked Electroencephalography (EEG) activity to capture 
neural processes associated with sensory, cognitive, and motor events1. These potentials 
reflect the summed, synchronous activity of large populations of cortical neurons, providing 
a millisecond-by-millisecond record of brain function.\\\\
The critical challenge in ERP analysis lies in overcoming the low signal-to-noise ratio, 
as the ERP is often obscured by higher-amplitude background EEG noise. To extract the signal, 
we employ the technique of \textbf{synchronized averaging}, relying on the premise that random 
noise cancels out over repeated trials, enhancing the time-locked ERP3.This laboratory session 
focuses on key ERP components elicited within choice reaction time tasks:

\begin{itemize}
    \item \textbf{Stimulus-Locked Potentials:} We analyze the $\textbf{P3 (P300)}$ component, a 
    marker of attention and working memory updating, typically studied here in the context of the 
    \textbf{Eriksen Flanker task}.
    \item \textbf{Response-Locked Potentials:} We evaluate the $\textbf{Error-Related Negativity (ERN)}$, 
    a component time-locked to an incorrect response, which serves as a physiological correlate of performance 
    monitoring and error detection.
\end{itemize}

The principal objectives of this practical work are structured as follows:
\begin{enumerate}
    \item Quantify the effect of \textbf{synchronized averaging} by examining how the \textbf{number of trials} 
    influences the stability of key ERP features (amplitude and latency).
    \item Assess the impact of poor event synchronization by simulating and analyzing the effect of 
    \textbf{misalignment} (time jitter) on the resulting average ERP7.
    \item Localize the $\textbf{P3}$ and $\textbf{ERN}$ generators by mapping their peak amplitude distributions 
    across the scalp using \textbf{topography}.
    \item Apply these techniques to a different dataset by analyzing the $\textbf{P300}$ from EEG recorded during 
    a \textbf{Visual Short Term Memory} task, complemented by a basic behavioral assessment.
\end{enumerate}