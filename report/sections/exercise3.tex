\section{Effect of Trials on ERP Feature Stability}

In this section, we quantify how the number of averaged trials influences the stability of the
Event-Related Potential (ERP) features.  
Specifically, we analyse two canonical components:

\begin{itemize}
    \item The \textbf{P3} (stimulus-locked), measured from incongruent trials.
    \item The \textbf{Error-Related Negativity (ERN)} (response-locked), measured from corrected responses.
\end{itemize}

Both amplitude and latency were computed for increasing numbers of averaged epochs,
ranging from $N=5$ to $N=200$ in steps of five.

The analysis was performed on three representative electrodes:

\[
\text{Fz (frontal)},\quad \text{Cz (central)},\quad \text{Pz (parietal)}.
\]

This selection captures the expected topographies of both components:  
P3 maximally expressed parietally, and ERN maximally expressed fronto-centrally.

% ==========================================================
% FIGURE: Amplitude and Latency vs Number of Trials
% ==========================================================

\begin{figure}[H]
    \centering
    \includegraphics[width=\textwidth]{img/Comp-AV-P3-ERN.png}
    \caption{Evolution of ERP feature extraction as a function of the number of averaged trials
    for channels Fz, Cz and Pz. Top-left: P3 amplitude. Top-right: ERN amplitude. 
    Bottom-left: P3 latency. Bottom-right: ERN latency.}
    \label{fig:erp_features_trials}
\end{figure}

% ==========================================================
% SUBSECTION: P3 RESULTS
% ==========================================================

\subsection{P3 Component (Stimulus-Locked)}

\subsubsection*{Amplitude}

A clear stabilisation pattern emerges (Fig.~\ref{fig:erp_features_trials}):  
with small numbers of trials ($N < 20$), the extracted amplitude fluctuates substantially.
This reflects the low signal-to-noise ratio typical of early-stage averaging.

As $N$ increases:

\begin{itemize}
    \item Fz shows the smallest P3 amplitude, consistent with its frontal topography.
    \item Cz increases moderately and becomes stable around $N \approx 60$-$80$.
    \item Pz shows the expected maximal amplitude with a robust peak between $N=20$-$40$.
\end{itemize}

This behaviour aligns with well-known characteristics of the P3b component, 
which peaks over parietal regions and requires sufficient averaging to overcome background EEG noise.

\subsubsection*{Latency}

P3 latency converges faster than amplitude.  
Although early averages ($N < 20$) produce unstable peak detection, all channels
stabilise at physiologically plausible latencies (approx.\ 350-380~ms) for
$N \geq 40$.

Notably, Pz again displays the most consistent values, reflecting its higher signal strength.

% ==========================================================
% SUBSECTION: ERN RESULTS
% ==========================================================

\subsection{ERN Component (Response-Locked)}

\subsubsection*{Amplitude}

The ERN amplitude behaves similarly:  
initially unstable when few trials are averaged, but progressively converging to
a characteristic negative deflection ($-10$ to $-15\,\mu\mathrm{V}$) as $N$ increases.

Topographically, the ERN is maximal at Fz and Cz — a hallmark of its medial frontal generators.
Pz exhibits a notably reduced amplitude, as expected.

Stability is largely reached with $N \geq 40$-$60$ trials.

\subsubsection*{Latency}

ERN latency stabilizes around 30 - 70 ms, consistent with the classical timing
of early error-monitoring processes originating from the anterior cingulate cortex.

The transition from unstable measurements ($N < 20$) to stable and reproducible latencies 
occurs again around $N \approx 40$-$50$.

% ==========================================================
% DISCUSSION — MINIMUM TRIALS
% ==========================================================

\subsection{Minimum Number of Trials Required for Stable ERP Features}

The results strongly support the notion that ERP components require adequate averaging
to overcome spontaneous EEG variability.

Across all channels and features:

\begin{itemize}
    \item The stabilisation of amplitude and latency consistently occurs between 
    \textbf{40 and 60 averaged trials}.
    \item This threshold is remarkably similar for both the P3 and ERN components.
    \item Components with stronger topographical expression (P3 at Pz, ERN at Fz/Cz) achieve
    stability slightly faster.
    \item With fewer than 20 trials, amplitudes and latencies exhibit large fluctuations,
    making the measurements unreliable for scientific or clinical interpretation.
\end{itemize}

\subsection{Interpretation}

The convergence analyses demonstrate that averaging is essential to recover clean, interpretable ERP
features. Both P3 and ERN follow classic neurophysiological patterns in amplitude and latency,
and require a minimum of 40-60 repetitions to produce robust and stable metrics. If we have to give a final number we would agree on 50 as a good threshold.

This finding is consistent with the literature on EEG signal processing and highlights the
trade-off between experiment duration and ERP feature reliability.