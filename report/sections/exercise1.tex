\section{Analysis of the First Five ERP Epochs}

\subsection*{Introduction}

The purpose of this first exercise is to examine the morphology and variability of 
the initial five EEG epochs recorded in both stimulus-locked and response-locked 
conditions. Event-Related Potentials (ERPs) are time-locked neural responses that 
reflect perceptual, cognitive, and motor processes. However, ERPs are embedded 
within the background EEG, which exhibits large stochastic fluctuations.  
Therefore, single-trial ERPs often do not resemble the canonical components 
typically observed after averaging many repetitions.

This section evaluates how five individual epochs behave across the midline 
electrodes Fz, Cz, and Pz, and demonstrates why averaging is essential for 
recovering P2, N2, P3, and ERN components.

% ============================================================
%                 Stimulus-Locked Figures
% ============================================================

\subsection{Stimulus-Locked Epochs (P2, N2, P3)}

\begin{figure}[H]
    \centering
    \includegraphics[width=0.8\textwidth]{img/ERP-Fz.png}
    \caption{Stimulus-locked single-trial epochs (first 5 trials) — Channel Fz. 
    Large fluctuations dominate the waveform, preventing identification 
    of early or late ERP components.}
    \label{fig:stim_fz}
\end{figure}

\begin{figure}[H]
    \centering
    \includegraphics[width=0.8\textwidth]{img/ERP-Cz.png}
    \caption{Stimulus-locked single-trial epochs (first 5 trials) — Channel Cz.
    Midline activity remains highly variable, masking morphology typically 
    associated with P2, N2, or P3.}
    \label{fig:stim_cz}
\end{figure}

\begin{figure}[H]
    \centering
    \includegraphics[width=0.8\textwidth]{img/ERP-Pz.png}
    \caption{Stimulus-locked single-trial epochs (first 5 trials) — Channel Pz.
    Parietal oscillations, including strong alpha rhythms, obscure stimulus-locked 
    components.}
    \label{fig:stim_pz}
\end{figure}

\subsubsection*{Interpretation of Stimulus-Locked Epochs}

Across the three midline electrodes (Fz, Cz, Pz), the first five stimulus-locked 
epochs exhibit substantial amplitude variability (commonly $\pm 30$--$40\,\mu$V), 
typical of raw EEG. Such variability masks the underlying ERP components:

\begin{itemize}
    \item \textbf{P2 (150–250 ms)} cannot be isolated due to the dominance of 
    high-frequency oscillations.
    \item \textbf{N2 (around 200 ms)} does not appear consistently across trials.
    \item \textbf{P3 (250–600 ms)} is not visually identifiable in any channel.
\end{itemize}

There are also no observable differences between congruent and incongruent 
conditions, which is expected because cognitive contrasts only emerge reliably 
after substantial averaging.That is going to be done in the next excercice.


\vspace{0.5cm}
\textbf{Topographically:}
\begin{itemize}
    \item \textbf{Fz} shows slow frontal drifts consistent with prefrontal sources.
    \item \textbf{Cz} presents fluctuating midline activity but no stable ERP features.
    \item \textbf{Pz} displays rhythmic parietal oscillatory activity, predominantly alpha,
    which masks potential late ERP components.
\end{itemize}

This confirms that single-trial recordings are insufficient to extract 
stimulus-locked ERP morphology.


% ============================================================
%                   Response-Locked Figures
% ============================================================

\subsection{Response-Locked Epochs (ERN)}

\begin{figure}[H]
    \centering
    \includegraphics[width=0.8\textwidth]{img/ERN-Fz.png}
    \caption{Response-locked single-trial epochs (first 5 trials) — Channel Fz.
    Although ERN typically peaks maximally at fronto-central sites, no identifiable 
    ERN component is visible at the single-trial level.}
    \label{fig:ern_fz}
\end{figure}

\begin{figure}[H]
    \centering
    \includegraphics[width=0.8\textwidth]{img/ERN-Cz.png}
    \caption{Response-locked single-trial epochs (first 5 trials) — Channel Cz.
    Strong fluctuations obscure the expected ERN (0--100 ms post-response).}
    \label{fig:ern_cz}
\end{figure}

\begin{figure}[H]
    \centering
    \includegraphics[width=0.8\textwidth]{img/ERN-Pz.png}
    \caption{Response-locked single-trial epochs (first 5 trials) — Channel Pz.
    Parietal areas exhibit weaker error-related activity and pronounced noise.}
    \label{fig:ern_pz}
\end{figure}

\subsubsection*{Interpretation of Response-Locked Epochs}

The response-locked windows show similar behaviour to the stimulus-locked case. 
The ERN is a small negative deflection occurring 0--100 ms after an erroneous 
response, typically ranging from $-5$ to $-15\,\mu$V. Because of its low 
amplitude, it is completely masked by single-trial noise:

\begin{itemize}
    \item No detectable ERN is visible in the first five trials in any channel.
    \item Fz and Cz, where ERN should be maximal, show considerable variability 
    and inconsistent polarity.
    \item Pz exhibits even weaker error-related signals, as expected given its 
    distance from the anterior cingulate cortex (ACC), the ERN generator.
\end{itemize}

These observations highlight that the ERN cannot be interpreted without averaging 
a sufficiently large number of error trials.


% ============================================================
%                 Global Interpretation
% ============================================================

\subsection{ Interpretation}

Taken together, the stimulus-locked and response-locked analyses show that 
single-trial EEG is dominated by spontaneous cortical activity, eye blinks, 
muscle artifacts, and non-phase-locked noise. As a consequence:

\begin{itemize}
    \item \textbf{ERP components are not visible} with only 5 trials.
    \item \textbf{Congruency effects are completely obscured}.
    \item \textbf{Topographic specializations} (frontal ERN, parietal P3) do not emerge.
    \item \textbf{Single-trial polarity is inconsistent}, making interpretation impossible.
\end{itemize}

From a neurophysiological standpoint, the absence of identifiable components in 
single epochs directly illustrates the necessity of synchronized averaging in ERP 
research. ERP components represent phase-locked neural activity, whereas noise is 
stochastic; therefore, only averaging cancels out uncorrelated fluctuations and 
reveals consistent time-locked responses.