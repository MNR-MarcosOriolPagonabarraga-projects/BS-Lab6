\section*{3.7 Topographic Distribution of P3 and ERN}

To localize the sources of the ERP components on the scalp, we analysed the spatial
distribution of the peak amplitudes across 19 standard electrode positions.
The P3 amplitude was calculated as the maximum positive deflection (250-600 ms)
for both congruent and incongruent stimuli. The ERN was quantified as the maximum
absolute negative deflection (0-100 ms) in the response-locked averages.

\begin{figure}[H]
    \centering
    \includegraphics[width=\textwidth]{img/TopographicDistributions.png} 
    \caption{Topographic distribution of ERP peaks. Left: P3 Congruent. Center: P3 Incongruent. Right: ERN (absolute amplitude).}
    \label{fig:topos}
\end{figure}

\subsection*{Interpretation}

\begin{itemize}
    \item \textbf{P3 Distribution:} As shown in the left and center topograms, the P3 component
    is maximally distributed over the \textbf{parietal and centro-parietal regions} (around Pz and Cz).
    This is consistent with the classic "P3b" topography associated with task-relevant
    stimulus processing. The amplitude appears slightly higher at the posterior area for incongruent stimuli
    (center) compared to congruent ones (left). 
    
    \item \textbf{ERN Distribution:} The ERN (right map) shows a distinct \textbf{fronto-central distribution}
    (maximal around Fz and FCz). This topography is distinct from the P3 and aligns
    perfectly with the anatomical location of the Anterior Cingulate Cortex (ACC),
    which is considered the neural generator of error monitoring signals.
\end{itemize}