\section{Exercise 2: Visual Short Term Memory (VSTM) Task}

In this section, we analyse EEG data recorded from a different cognitive paradigm:
a Visual Short Term Memory task (Student 2 data).
We extracted the stimulus-locked P300 component elicited by the second image of the pair,
considering only trials with correct responses. The signals were low-pass filtered (7 Hz) to smooth the waveforms.

\subsection{P300 Waveforms and Topography (Student 2)}

\begin{figure}[H]
    \centering
    \includegraphics[width=0.9\textwidth]{img/VSTM_ERPs.png}
    \caption{Stimulus-locked P300 for Student 2 at Fz, Cz, and Pz. Blue: Congruent. Red: Incongruent. Dashed: Joint Average. Note the high variability and oscillatory nature of the signals.}
    \label{fig:vstm_erps}
\end{figure}

\begin{figure}[H]
    \centering
    \includegraphics[width=\textwidth]{img/VSTM_Topos.png}
    \caption{Topographic distribution of the P300 peak (250-400 ms) for the VSTM task. The distribution is atypical, showing strong lateralization.}
    \label{fig:vstm_topos}
\end{figure}

\subsubsection*{Interpretation}

The waveforms in Figure \ref{fig:vstm_erps} display considerable variability compared to the standard Grand Averages seen in Exercise 1. 
\begin{itemize}
    \item \textbf{Waveforms:} At \textbf{Pz} and \textbf{Fz}, the Congruent condition (blue) shows a distinct positive peak around 350--400 ms, consistent with P300 timing. However, the Incongruent condition (red) shows a different morphology, with a delayed or broader positivity particularly visible at \textbf{Cz} around 450 ms.
    \item \textbf{Topography:} The topographic maps (Figure \ref{fig:vstm_topos}) deviate from the classic midline centro-parietal distribution. The \textbf{Congruent} condition shows a focal positivity over the left central/parietal region, while the \textbf{Incongruent} condition displays positivity concentrated in the right posterior/occipital region. The \textbf{Joint} average shows a bilateral posterior distribution.
\end{itemize}
These irregular distributions likely reflect the lower signal-to-noise ratio of a single-subject recording (Student 2) compared to the grand average of 9 subjects used in the previous exercise.

\subsection{Behavioral Analysis}

We calculated the response time (RT) and accuracy for the congruent (matching images)
and incongruent (non-matching images) conditions.

\begin{table}[H]
\centering
\begin{tabular}{|l|c|c|}
\hline
\textbf{Condition} & \textbf{Response Time (Mean $\pm$ STD)} & \textbf{Accuracy (\%)} \\
\hline
Congruent   & $0.147 \pm 0.033$ s & 96\% \\ 
Incongruent & $0.145 \pm 0.034$ s & 100\% \\ 
\hline
\end{tabular}
\caption{Behavioral performance for Student 2.}
\label{tab:behavior}
\end{table}

The behavioral results for this specific student are notably fast ($\approx 145$ ms). 
Contrary to the standard "Stroop-like" interference effect—where incongruent stimuli 
typically cause slower reactions and more errors—this subject performed \textbf{equally fast (or slightly faster)} 
on incongruent trials and achieved \textbf{100\% accuracy}. This suggests the subject did not experience the 
expected cognitive conflict, or the task difficulty was low enough to allow ceiling performance.