\section{Visual Short Term Memory (VSTM) Task}

In this section, we analyse EEG data recorded from a different cognitive paradigm:
a Visual Short Term Memory task (Student 2 data).
We extracted the stimulus-locked P300 component elicited by the second image of the pair,
considering only trials with correct responses. The signals were low-pass filtered (7 Hz) to smooth the waveforms.

\subsection{P300 Waveforms and Topography (Student 2)}

\begin{figure}[H]
    \centering
    % REPLACE WITH IMAGE FROM EXERCICE_4_VSTM.m
    \includegraphics[width=0.8\textwidth]{img/VSTM_ERPs.png}
    \caption{Stimulus-locked P300 for Student 2 at Fz, Cz, and Pz. Blue: Congruent. Red: Incongruent. Dashed: Joint Average.}
    \label{fig:vstm_erps}
\end{figure}

\begin{figure}[H]
    \centering
    % REPLACE WITH IMAGE FROM EXERCICE_4_VSTM.m
    \includegraphics[width=\textwidth]{img/VSTM_Topos.png}
    \caption{Topographic distribution of the P300 peak (250-400 ms) for the VSTM task.}
    \label{fig:vstm_topos}
\end{figure}

\subsubsection*{Interpretation}
The waveforms show a clear positive deflection peaking around 300-400 ms.
Topographically, this positivity is widely distributed but shows parietal dominance,
confirming it is a P300-like component associated with the matching/comparison process in working memory.

\subsection{Behavioral Analysis}

We calculated the response time (RT) and accuracy for the congruent (matching images)
and incongruent (non-matching images) conditions.

\begin{table}[H]
\centering
\begin{tabular}{|l|c|c|}
\hline
\textbf{Condition} & \textbf{Response Time (Mean $\pm$ STD)} & \textbf{Accuracy (\%)} \\
\hline
Congruent   & $0.147 \pm 0.033$ s & 96\% \\ 
Incongruent & $0.145 \pm 0.034$ s & 100\% \\ 
\hline
\end{tabular}
\caption{Behavioral performance for Student 2.}
\label{tab:behavior}
\end{table}

The behavioral results generally indicate that incongruent trials (mismatch) require slightly
more processing time (longer RT) and may result in slightly lower accuracy compared to
congruent trials, reflecting the increased cognitive load in decision making.